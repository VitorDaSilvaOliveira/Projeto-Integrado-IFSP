%% abtex2-modelo-trabalho-academico.tex, v-1.9.7 laurocesar
%% Copyright 2012-2018 by abnTeX2 group at http://www.abntex.net.br/ 
%%
%% This work may be distributed and/or modified under the
%% conditions of the LaTeX Project Public License, either version 1.3
%% of this license or (at your option) any later version.
%% The latest version of this license is in
%%   http://www.latex-project.org/lppl.txt
%% and version 1.3 or later is part of all distributions of LaTeX
%% version 2005/12/01 or later.
%%
%% This work has the LPPL maintenance status `maintained'.
%% 
%% The Current Maintainer of this work is the abnTeX2 team, led
%% by Lauro César Araujo. Further information are available on 
%% http://www.abntex.net.br/
%%
%% This work consists of the files abntex2-modelo-trabalho-academico.tex,
%% abntex2-modelo-include-comandos and abntex2-modelo-references.bib
%%

% ------------------------------------------------------------------------
% ------------------------------------------------------------------------
% abnTeX2: Modelo de Trabalho Academico (tese de doutorado, dissertacao de
% mestrado e trabalhos monograficos em geral) em conformidade com 
% ABNT NBR 14724:2011: Informacao e documentacao - Trabalhos academicos -
% Apresentacao
% ------------------------------------------------------------------------
% ------------------------------------------------------------------------

\documentclass[
	% -- opções da classe memoir --
	12pt,				% tamanho da fonte
	openany,			% capítulos começam em pág ímpar (insere página vazia caso preciso)
	twoside,			% para impressão em recto e verso. Oposto a oneside
	a4paper,			% tamanho do papel. 
	% -- opções da classe abntex2 --
	%chapter=TITLE,		% títulos de capítulos convertidos em letras maiúsculas
	%section=TITLE,		% títulos de seções convertidos em letras maiúsculas
	%subsection=TITLE,	% títulos de subseções convertidos em letras maiúsculas
	%subsubsection=TITLE,% títulos de subsubseções convertidos em letras maiúsculas
	% -- opções do pacote babel --
	english,			% idioma adicional para hifenização
	brazil				% o último idioma é o principal do documento
	]{abntex2}

% ---
% Pacotes básicos 
% ---
\usepackage{lmodern}			% Usa a fonte Latin Modern			
\usepackage[T1]{fontenc}		% Selecao de codigos de fonte.
\usepackage[utf8]{inputenc}		% Codificacao do documento (conversão automática dos acentos)
\usepackage{indentfirst}		% Indenta o primeiro parágrafo de cada seção.
\usepackage{color}				% Controle das cores
\usepackage{graphicx}			% Inclusão de gráficos
\usepackage{microtype} 			% para melhorias de justificação
\usepackage{placeins}


% ---
		
% ---
% Pacotes adicionais, usados apenas no âmbito do Modelo Canônico do abnteX2
% ---
\usepackage{lipsum}				% para geração de dummy text
% ---

% ---
% Pacotes de citações
% ---
\usepackage[brazilian,hyperpageref]{backref}	 % Paginas com as citações na bibl
\usepackage[alf]{abntex2cite}	% Citações padrão ABNT
\usepackage{hyperref}
\usepackage{xurl} % quebra URLs longas automaticamente
% --- 
% CONFIGURAÇÕES DE PACOTES
% --- 

% ---
% Configurações do pacote backref
% Usado sem a opção hyperpageref de backref
\renewcommand{\backrefpagesname}{Citado na(s) página(s):~}
% Texto padrão antes do número das páginas
\renewcommand{\backref}{}
% Define os textos da citação
\renewcommand*{\backrefalt}[4]{
	\ifcase #1 %
		Nenhuma citação no texto.%
	\or
		Citado na página #2.%
	\else
		Citado #1 vezes nas páginas #2.%
	\fi}%
% ---

% ---
% Informações de dados para CAPA e FOLHA DE ROSTO
% ---
\titulo{VIP PENHA}
\autor{
  {\large IFSP - Instituto Federal de Educação, Ciência e Tecnologia} \\
  {\large Câmpus São Paulo} \\[1cm]
  \makebox[0.60\textwidth][l]{BEATRIZ MUNIZ DE BARROS} \makebox[0.20\textwidth][r]{SP3161315}\\
  \makebox[0.60\textwidth][l]{GEAN CARLOS DE SOUSA BANDEIRA} \makebox[0.20\textwidth][r]{SP3030075}\\
  \makebox[0.60\textwidth][l]{KHALIL KHALID ABOU ANCHE} \makebox[0.20\textwidth][r]{SP3121925}\\
  \makebox[0.60\textwidth][l]{MARCELO FLORES VALDEZ} \makebox[0.20\textwidth][r]{SP3039056}\\
  \makebox[0.60\textwidth][l]{MATHEUS PRANDO APPOLINARIO BARBOSA} \makebox[0.20\textwidth][r]{SP3121747}\\
  \makebox[0.60\textwidth][l]{RAFAEL VALVERDE ZANATA DA SILVA} \makebox[0.20\textwidth][r]{SP3119866}\\
  \makebox[0.60\textwidth][l]{VITOR DA SILVA OLIVEIRA} \makebox[0.20\textwidth][r]{SP3020589}\\
}


\local{São Paulo - SP - Brasil}
\data{2025}
\orientador{Marcelo Tavares de Santana}
\instituicao{
  IFSP - Instituto Federal de Educação, Ciência e Tecnologia \\
  Câmpus São Paulo \\
  Tecnologia em Análise e Desenvolvimento de Sistemas
}
\tipotrabalho{Projeto Integrado I}
\preambulo{
 Projeto desenvolvido como parte das atividades da disciplina Projeto Integrado, apresentado ao Instituto Federal de Educação, Ciência e Tecnologia de São Paulo, Câmpus São Paulo, curso de Tecnologia em Análise e Desenvolvimento de Sistemas.
}
% ---


% ---
% Configurações de aparência do PDF final

% alterando o aspecto da cor azul
\definecolor{blue}{RGB}{41,5,195}

% informações do PDF
\makeatletter
\hypersetup{
     	%pagebackref=true,
		pdftitle={\@title}, 
		pdfauthor={\@author},
    	pdfsubject={\imprimirpreambulo},
	    pdfcreator={LaTeX with abnTeX2},
		pdfkeywords={abnt}{latex}{abntex}{abntex2}{trabalho acadêmico}, 
		colorlinks=true,       		% false: boxed links; true: colored links
    	linkcolor=blue,          	% color of internal links
    	citecolor=blue,        		% color of links to bibliography
    	filecolor=magenta,      		% color of file links
		urlcolor=blue,
		bookmarksdepth=4
}
\makeatother
% --- 

% ---
% Posiciona figuras e tabelas no topo da página quando adicionadas sozinhas
% em um página em branco. Ver https://github.com/abntex/abntex2/issues/170
\makeatletter
\setlength{\@fptop}{5pt} % Set distance from top of page to first float
\makeatother
% ---

% ---
% Possibilita criação de Quadros e Lista de quadros.
% Ver https://github.com/abntex/abntex2/issues/176
%
\newcommand{\quadroname}{Quadro}
\newcommand{\listofquadrosname}{Lista de quadros}

\newfloat[chapter]{quadro}{loq}{\quadroname}
\newlistof{listofquadros}{loq}{\listofquadrosname}
\newlistentry{quadro}{loq}{0}

% configurações para atender às regras da ABNT
\setfloatadjustment{quadro}{\centering}
\counterwithout{quadro}{chapter}
\renewcommand{\cftquadroname}{\quadroname\space} 
\renewcommand*{\cftquadroaftersnum}{\hfill--\hfill}

\setfloatlocations{quadro}{hbtp} % Ver https://github.com/abntex/abntex2/issues/176

% Evitar página em branco após o resumo
\let\oldcleardoublepage\cleardoublepage
\let\cleardoublepage\clearpage
% ---

% --- 
% Espaçamentos entre linhas e parágrafos 
% --- 

% O tamanho do parágrafo é dado por:
\setlength{\parindent}{1.3cm}

% Controle do espaçamento entre um parágrafo e outro:
\setlength{\parskip}{0.2cm}  % tente também \onelineskip

% ---
% compila o indice
% ---
\makeindex
% ---

% ----
% Início do documento
% ----
\begin{document}

% Seleciona o idioma do documento (conforme pacotes do babel)
%\selectlanguage{english}
\selectlanguage{brazil}

% Retira espaço extra obsoleto entre as frases.
\frenchspacing 

% ----------------------------------------------------------
% ELEMENTOS PRÉ-TEXTUAIS
% ----------------------------------------------------------
% \pretextual

% ---
% Capa
% ---



% ---

% ---
% Folha de rosto
% (o * indica que haverá a ficha bibliográfica)
\begin{capa}

  \begin{center}
    {\ABNTEXchapterfont\large\imprimirautor}

    \vspace*{\fill}\vspace*{\fill}
    \begin{center}
      \ABNTEXchapterfont\bfseries\Large\imprimirtitulo
    \end{center}
    \vspace*{\fill}
    
    \hspace{.45\textwidth}
    \begin{minipage}{.5\textwidth}
    \end{minipage}%

    \vspace*{1.5cm} % Aumenta o espaço após o preâmbulo

    \vspace*{0.8cm} % Diminui o espaço antes da data

  \end{center}

  \begin{center}
    {\large\imprimirlocal}
    \par
    {\large\imprimirdata}
  \end{center}

\end{capa}

\begin{folhadeaprovacao}

  \begin{center}
    {\ABNTEXchapterfont\large\imprimirautor}

    \vspace*{\fill}\vspace*{\fill}
    \begin{center}
      \ABNTEXchapterfont\bfseries\Large\imprimirtitulo
    \end{center}
    \vspace*{\fill}
    
    \hspace{.45\textwidth}
    \begin{minipage}{.5\textwidth}
        \imprimirpreambulo
    \end{minipage}%

    \vspace*{1.5cm} % Aumenta o espaço após o preâmbulo

    \begin{center}
        IFSP - Instituto Federal de Educação, Ciência e Tecnologia \\
        Câmpus São Paulo \\
        Tecnologia em Análise e Desenvolvimento de Sistemas
    \end{center}

    \vspace*{0.8cm} % Diminui o espaço antes da data

  \end{center}

  \begin{center}
    {\large\imprimirlocal}
    \par
    {\large\imprimirdata}
  \end{center}

\end{folhadeaprovacao}

% ---

% ---
% Dedicatória
% ---

% ---


% ---
% RESUMOS
% ---


% resumo em português

\setlength{\absparsep}{18pt} % ajusta o espaçamento dos parágrafos do resumo
\begin{resumo}
Este projeto descreve a criação de um sistema de controle de estoque para a loja de eletrônicos VIP PENHA. Em um cenário de rápido avanço tecnológico e crescente competitividade de mercado, a gestão eficiente de recursos é essencial para empreendimentos. O sistema foi desenvolvido para otimizar o controle de produtos, tornando a operação mais eficiente. 

A problemática central é a falta de controle de estoque e os erros de processos manuais na "VIP PENHA". A solução implementada é um sistema unificado que permite o cadastro de produtos, o controle de entradas e saídas de mercadorias, e a geração automática de relatórios gerenciais. Isso resultou na redução de erros manuais, na diminuição do tempo gasto na gestão do estoque e na melhoria da tomada de decisões estratégicas, oferecendo uma perspectiva clara sobre o desempenho do inventário. 

Para o desenvolvimento do sistema, foram realizadas pesquisas sobre gestão de estoque e sistemas de informação, utilizando uma metodologia de trabalho ágil, o Kanban. O projeto abrangeu o detalhamento de funcionalidades, requisitos técnicos, arquitetura do sistema (front-end, back-end, banco de dados e infraestrutura), tecnologias empregadas, planos de testes e planejamentos financeiros.


 \textbf{Palavras-chave}: Sistema de estoque. Gestão de produtos. Automação. Relatórios gerenciais. Kanban. 
\end{resumo}

% resumo em inglês
\begin{resumo}[Abstract]
 \begin{otherlanguage*}{english}
This project describes the creation of an inventory control system for the VIP PENHA electronics store. In a scenario of rapid technological advancement and increasing market competitiveness, efficient resource management is essential for businesses. The system was developed to
optimize product control, making operations more efficient. 

The central problem addressed is the lack of inventory control and manual process errors at "VIP PENHA". The implemented solution is a unified system that allows product registration, control of goods entry and exit, and automatic generation of management reports. This resulted in the reduction of manual errors, decreased time spent on inventory management, and improved strategic decision-making, offering a clear perspective on inventory performance. 

For the system's development, research on inventory management and information systems was conducted, utilizing an agilemethodology, Kanban. The project covered the detailing of functionalities, technical requirements, system architecture (front-end, back-end, database, and infrastructure), technologies employed, test plans, and financial planning.


   \vspace{\onelineskip}
 
   \noindent 
   \textbf{Keywords}: Inventory management system. Product management. Automation. Management reports. Kanban.
 \end{otherlanguage*}
\end{resumo}


% ---

% ---
% inserir lista de ilustrações
% ---
\pdfbookmark[0]{\listfigurename}{lof}
\listoffigures*
\clearpage
% ---

% ---
% inserir lista de quadros
% ---
\pdfbookmark[0]{\listofquadrosname}{loq}
\listofquadros*
\cleardoublepage
% ---

% ---
% inserir lista de tabelas
% ---
\pdfbookmark[0]{\listtablename}{lot}
\listoftables*
\cleardoublepage
% ---

% --
% inserir lista de símbolos
% ---

% ---

% ---
% inserir o sumario
% ---
\pdfbookmark[0]{\contentsname}{toc}
\tableofcontents*
\cleardoublepage
% ---



% ----------------------------------------------------------
% ELEMENTOS TEXTUAIS
% ----------------------------------------------------------
\textual

% ----------------------------------------------------------
% Introdução (exemplo de capítulo sem numeração, mas presente no Sumário)
% ----------------------------------------------------------
\chapter{Introdução}
% ----------------------------------------------------------
Em um mercado cada vez mais competitivo, a gestão eficiente de recursos tornou-se essencial para o sucesso das empresas, especialmente em lojas de artigos eletrônicos como a VIP PENHA. O setor de eletrônicos tem crescido de forma constante, impulsionado pelo avanço tecnológico e pela alta demanda por dispositivos inteligentes, acessórios e equipamentos de informática. De acordo com dados da \citeonline{ABINEE2024}, o mercado brasileiro de eletrônicos movimenta bilhões de reais por ano, sendo um dos segmentos mais dinâmicos e inovadores da economia. Esse crescimento, entretanto, também traz desafios, principalmente para pequenas e médias empresas, que precisam lidar com grande variedade de produtos, constantes atualizações tecnológicas e margens de lucro cada vez mais apertadas.

A transformação digital tem impulsionado mudanças significativas no ambiente corporativo, exigindo das organizações a adoção de práticas tecnológicas que aprimorem seu desempenho. Ferramentas digitais possibilitam operações mais estruturadas, reduzem falhas e embasam decisões estratégicas por meio de dados confiáveis, tornando-se um diferencial competitivo indispensável. Segundo \citeonline{laudon2014}, os sistemas de informação constituem a base para conduzir os negócios na era atual, permitindo às empresas alcançar excelência operacional e desenvolver novos produtos e serviços.

No caso específico da gestão de estoque, a integração tecnológica garante maior visibilidade do fluxo de produtos, minimiza erros operacionais e perdas de informação, além de permitir um planejamento mais eficiente de reposição e armazenamento. O acesso a dados precisos e atualizados fortalece o processo de decisão e amplia a capacidade de expansão da empresa. Para uma loja como a VIP PENHA, que trabalha com produtos de diferentes categorias e faixas de preço, a organização do estoque é um fator determinante para manter a disponibilidade dos itens e evitar prejuízos com produtos parados ou desatualizados.

Além disso, o uso de um sistema de gerenciamento de estoque contribui diretamente para a melhoria do atendimento ao cliente, pois permite prever demandas sazonais, identificar os produtos mais vendidos e antecipar tendências de consumo. Isso possibilita que a empresa mantenha um portfólio de produtos mais alinhado às preferências do público, aumentando sua competitividade no mercado local. 

Dessa forma, a digitalização deixou de ser uma opção e passou a representar um pilar estratégico para empresas que buscam eficiência, segurança e competitividade. A implementação de um sistema de gerenciamento de estoque na VIP PENHA alinha-se a essa realidade, contribuindo para a sustentabilidade e o crescimento do negócio diante das demandas do mercado moderno

\section{Objetivos}

Nesta seção serão apresentados os objetivos do projeto, ele estão divididos em objetivo geral e objetivos específicos. O objetivo geral descreve a meta principal do projeto, já os objetivos específicos descrevem as etapas e funcionalidades para o alcance da meta final.

\subsection{Objetivo Geral}

Auxiliar a gestão da loja VIP Penha por meio do desenvolvimento de um sistema de gerenciamento de estoque para a loja VIP PENHA, visando transformar a gestão de inventário em um processo automatizado e estratégico para otimizar o controle de produtos, diminuir erros manuais e reduzir o tempo gasto na gestão, ao possibilitar o cadastro detalhado de itens, o monitoramento de entradas e saídas em tempo real e a configuração de alertas de estoque mínimo. 

\subsection{Objetivos Específicos}

Para alcançar o objetivo geral proposto, torna-se necessário estabelecer metas específicas que orientem o desenvolvimento do sistema de forma organizada e funcional. Esses objetivos buscam garantir que a solução proposta atenda às principais demandas de controle, eficiência e confiabilidade na gestão de estoque da empresa.

\begin{itemize}
    \item Desenvolver uma plataforma de cadastro de produtos, permitindo o registro e atualização dos produtos no estoque.
    \item Permitir a análise do desempenho do estoque por meio de relatórios gerenciais, identificando produtos mais vendidos, vendas realizadas e necessidades de reposição.
    \item Reduzir os erros manuais ao implementar processos automatizados.
    \item Facilitar a tomada de decisão através de indicadores e gráficos que permitam visualizar o comportamento das vendas e a movimentação dos produtos.
    \item Garantir a integridade e segurança das informações, assegurando que os dados cadastrados e relatórios gerados sejam armazenados e acessados de forma confiável.
    \item Proporcionar uma interface intuitiva e acessível, de modo que usuários com diferentes níveis de familiaridade tecnológica consigam operar o sistema sem dificuldades.
\end{itemize}

\section{Problema e Solução Proposta}

Nessa seção serão apresentados os problemas enfrentados pela loja VIP PENHA em relação ao gerenciamento do seu estoque, assim como a solução proposta para resolver esses problemas. A seguir, são descritos os principais desafios e como o sistema visa solucioná-los de maneira eficiente.

\subsection{Problema}

Devido a recentes expansões, a loja VIP PENHA vem enfrentando problemas como a falta de controle do seu estoque devido a ausência de um sistema automatizado. Essa lacuna tem gerado uma série de complicações que comprometem o funcionamento eficiente da loja. A principal delas é a falta de controle em tempo real das mercadorias, dificultando o acompanhamento preciso da quantidade de produtos disponíveis e resultando em faltas ou excessos de estoque.

Além disso, o processo manual de registro das entradas e saídas de produtos tem sido fonte constante de erros, como registros duplicados, extravios e divergências entre o estoque físico e o registrado. A inexistência de relatórios gerenciais confiáveis impede uma análise clara do desempenho da loja, dificultando a tomada de decisões estratégicas.

Outro ponto preocupante é a falta de rastreabilidade do histórico de movimentações dos produtos, o que inviabiliza a realização de auditorias e a identificação de padrões de consumo. Além disso, a ausência de integração entre os setores de compras, vendas e estoque gera desorganização e atrasos operacionais. Todos esses fatores combinados impactam negativamente a eficiência, o atendimento ao cliente e o potencial de crescimento da loja.


\subsection{Solução Proposta}

Para resolver esses problemas, propõe-se o desenvolvimento de um sistema de gerenciamento de estoque voltado às necessidades específicas da loja VIP PENHA. A implementação desse sistema permitirá centralizar e automatizar os principais processos envolvidos no controle de mercadorias, trazendo mais eficiência, organização e confiabilidade às operações da empresa.

O sistema contará com funcionalidades essenciais como o cadastro de produtos, controle automatizado das entradas e saídas de mercadorias e geração de relatórios gerenciais. Com isso, será possível acompanhar em tempo real o status do estoque, identificar rapidamente produtos em falta ou com excesso, e tomar decisões mais claramente com base nos relatórios fornecidos pelo sistema.

Diferente das soluções genéricas disponíveis no mercado, o sistema VIP PENHA será totalmente adaptado à realidade e ao fluxo de trabalho da loja, oferecendo uma interface simples, intuitiva e de fácil uso. Além disso, o sistema terá baixo custo de implementação e manutenção, tornando-se uma alternativa acessível e eficiente para pequenos comércios que desejam modernizar sua gestão de estoque sem recorrer a plataformas complexas e de alto custo.

Com a adoção dessa solução, a loja VIP PENHA poderá reduzir drasticamente os erros operacionais, melhorar o fluxo de trabalho, aumentar a produtividade da equipe e oferecer um atendimento mais ágil e eficiente aos clientes. Dessa forma, o sistema contribuirá significativamente para o crescimento sustentável e a consolidação da loja no mercado.
 

\section{Justificativa}
No ambiente de negócios atual, a gestão de estoque deixou de ser uma tarefa puramente operacional para se tornar um pilar estratégico, essencial para a sobrevivência e competitividade de qualquer empresa varejista. Conforme aponta \citeonline{ballou2006} em seus estudos sobre logística empresarial, o controle preciso do inventário impacta diretamente os custos, o nível de serviço ao cliente e, por consequência, a lucratividade do negócio. Ignorar a modernização desta área significa expor a empresa a riscos financeiros e operacionais que podem comprometer sua sustentabilidade.


Para dimensionar a urgência dessa modernização, é crucial analisar o panorama das micro e pequenas empresas no Brasil. Segundo o estudo Mapa de Digitalização das Micro e Pequenas Empresas Brasileiras de 2024, desenvolvido pela Fundação Getúlio Vargas (FGV) em parceria com a  \citeonline{ABDIMaturidade2024}, o índice médio de maturidade digital dos pequenos negócios é de 35 pontos, em uma escala de 0 a 80, indicando um nível de 43,75\% de maturidade média. A pesquisa aponta também que apenas 27\% das empresas possuem um sistema de gestão que integra as bases de dados de todas as áreas do negócio . 

Nesse cenário, a implementação de um sistema de gestão digital na loja VIP PENHA representa uma ação estratégica, não apenas para acompanhar a modernização do mercado, mas principalmente para solucionar gargalos que hoje prejudicam o desempenho e o crescimento do negócio. Na prática, a implementação deste sistema gera impactos positivos e mensuráveis na operação, destacando-se:

\begin{itemize}
    \item \textbf{Redução de erros operacionais:} O sistema eliminará divergências entre o estoque físico e o digital, garantindo registros precisos e confiáveis.
    \item \textbf{Acompanhamento do sistema:} Com dados digitalizados, é possível acompanhar todas as movimentações de entradas e saídas de produtos com uma extrema facilidade e confiabilidade.
    \item \textbf{Agilidade nos processos:} As movimentações de produtos serão registradas com mais velocidade e precisão, reduzindo o tempo gasto com conferências e atualizações manuais.
    \item \textbf{Otimização de recursos:} Com um sistema de gerênciamento tecnológico, é possível fazer uma melhor utilização do espaço físico disponível, além de um melhor controle dos recursos financeiros, evitando o excesso e a falta de produtos.
    \item \textbf{Geração de relatórios:} As informações poderão ser analisadas rapidamente por meio de relatórios personalizados, auxiliando na tomada de decisões estratégicas.
\end{itemize}

Digitalizar o sistema de estoque é, portanto, um passo essencial para empresas que desejam crescer de forma organizada, segura e competitiva. Ao automatizar esse setor, os gestores têm mais controle e previsibilidade, elementos indispensáveis para a sobrevivência e sustentabilidade de qualquer negócio, especialmente no cenário atual de constantes mudanças e demandas cada vez mais dinâmicas.

\section{Análise de Concorrência}

Nesta seção, foi realizada uma análise dos principais sistemas de gerenciamento de estoque disponíveis no mercado, com foco em soluções utilizadas por lojas de pequeno e médio porte. Assim, demostrando quais as vantagens de usar o sistema que produzimos.

\subsection{Concorrente 1: Bling ERP}
O Bling é um sistema ERP completo que oferece controle de estoque, vendas, emissão de notas fiscais e integração com plataformas de e-commerce. É bastante utilizado por empresas que também vendem online, oferecendo funcionalidades robustas. No entanto, seu uso pode ser complexo para iniciantes, além de exigir pagamento mensal.

\subsection{Concorrente 2: Tiny ERP}
O Tiny ERP oferece funcionalidades similares ao Bling, como controle de estoque, pedidos, emissão de notas fiscais e integração com o setor financeiro. É conhecido por sua interface amigável, mas ainda assim exige uma curva de aprendizado e também é um serviço pago.

\subsection{Concorrente 3: Nex}
O Nex é um sistema gratuito e simples, ideal para pequenos comércios. Permite o cadastro de produtos, controle de estoque e de vendas. É bastante intuitivo, mas possui limitações em relação à integração com outras plataformas e funcionalidades avançadas.

\subsection{Concorrente 4: MarketUP}
O MarketUP é uma solução gratuita e bastante completa, oferecendo controle de estoque, vendas, financeiro e emissão de notas fiscais. No entanto, a interface pode ser confusa, especialmente para usuários menos experientes, e o suporte técnico é limitado.

\subsection{Quadro Comparativo}

\begin{quadro}[htb]
\centering
\caption{\label{quadro_comparativo}Comparação entre Sistemas de Gerenciamento de Estoque}
\begin{tabular}{|p{3.2cm}|p{5.5cm}|p{2.2cm}|p{4.1cm}|}
\hline
\textbf{Sistema} & \textbf{Funcionalidades Principais} & \textbf{Preço} & \textbf{Observações} \\
\hline
\textbf{Bling ERP} & Controle de estoque, vendas, emissão de notas fiscais, integração com e-commerce & Pago & Funcional, mas complexo para iniciantes \\
\hline
\textbf{Tiny ERP} & Estoque, pedidos, notas fiscais, controle financeiro & Pago & Interface moderna, porém exige curva de aprendizado \\
\hline
\textbf{Nex} & Cadastro de produtos, estoque e vendas & Gratuito & Intuitivo, ideal para pequenos comércios, porém limitado \\
\hline
\textbf{MarketUP} & Estoque, vendas, financeiro, notas fiscais & Gratuito & Completo, mas com interface confusa e suporte limitado \\
\hline
\textbf{"Sistema VIP PENHA} & Controle de estoque em tempo real, alerta de estoque mínimo, cadastro técnico de produtos, controle de garantias e geração de relatórios personalizados & Gratuito/Personalizado & Sistema desenvolvido especificamente para a realidade da loja VIP PENHA, com interface simples e fácil aprendizado, garantindo maior eficiência e redução de erros. \\
\hline
\end{tabular}
\end{quadro}

% ---
% Capitulo com exemplos de comandos inseridos de arquivo externo 
% ---
\include{abntex2-modelo-include-comandos}
% ---

\chapter{Revisão da Literatura}
O presente capítulo tem como objetivo apresentar estudos, teorias e contribuições acadêmicas relacionados à gestão de estoque em empresas.
\section{Histórico do Gerenciamento de Estoque}
A gestão de estoque é uma prática que acompanha a humanidade há milênios, realizar uma armazenagem inteligente dos recursos se mostrou essencial para a raça humana desde o seu primórdio. Um exemplo é o Período Uruk, nele foram desenvolvidas várias técnicas de gestão, como o uso de imagens e símbolos para administrar a estocagem de grãos, frutas e produtos, o que impulsionou essa sociedade a grandes avanços e ao desenvolvimento de um dos primeiros sistemas de escrita da história \cite{Albright1996}.

Durante a revolução industrial, com a produção em larga escala, houve um grande aumento na necessidade de melhores práticas de gerenciamento de estoque. O aumento da demanda de abastecimento contínuo do mercado levou ao desenvolvimento de melhores técnicas de controle e armazenamento dos produtos \cite{DIAS2021}.

Em meados do século XX, com o avanço da computação e o surgimento de sistemas informatizados, ocorreu uma verdadeira revolução na gestão de estoque. O desenvolvimento de softwares específicos para administração de materiais, como o Material Requirements Planning (MRP), criado na década de 1960, permitiu que empresas passassem a planejar suas necessidades de produção com base em dados de consumo e previsão de demanda. Esse modelo evoluiu posteriormente para o Manufacturing Resource Planning (MRP II), que integrou outras áreas da empresa, como produção, compras e finanças, ampliando o controle sobre os processos internos \cite{slack2013}.

Já nos anos 1990, surge o Enterprise Resource Planning (ERP), sistema que consolidou os conceitos anteriores e os ampliou para toda a organização. O ERP unificou informações de diferentes setores em uma única plataforma, permitindo maior integração entre compras, estoque, vendas e contabilidade. Essa inovação marcou uma nova era na administração de estoques, com base em dados, indicadores e processos informatizados. Paralelamente, práticas como o Just in Time (JIT), desenvolvidas inicialmente no Japão pela Toyota, também influenciaram o modo como as empresas passaram a lidar com seus estoques, priorizando a redução de desperdícios e a reposição conforme a necessidade real de produção \cite{ohno1997}.

Nas décadas seguintes, com o avanço da globalização e da tecnologia da informação, a gestão de estoques passou a incorporar conceitos de cadeia de suprimentos (supply chain management), conectando empresas, fornecedores e clientes em tempo real. As ferramentas digitais se tornaram indispensáveis para lidar com grandes volumes de dados, prever demandas e otimizar o fluxo de mercadorias.

Dessa forma, o percurso histórico ressalta a complexidade e a necessidade de soluções de gestão de estoque modernas, capazes de integrar processos e informações de maneira eficiente. A evolução dos métodos, desde o controle manual até os sistemas ERP, demonstra que a tecnologia tornou-se essencial para a administração eficaz dos estoques. No entanto, também evidencia que muitas das soluções existentes no mercado não são adequadas à realidade de pequenos comércios, por demandarem altos investimentos e infraestrutura avançada.

Sendo assim, o desenvolvimento do sistema proposto para a VIP PENHA se apoia nesse entendimento histórico e tecnológico: trata-se de uma solução digital sob medida, que busca aliar os princípios da gestão moderna, precisão, integração e às necessidades práticas de um negócio local. 

\section{Atualidades do Gerenciamento de Estoque}
Atualmente, o processo de gerenciamento de estoque encontra-se em níveis elevados de integração tecnológica, com diversas ferramentas modernas desempenhando papéis fundamentais nessa tarefa. A utilização de sistemas automatizados, softwares de gestão integrada (ERP) e tecnologias de rastreamento tem sido amplamente adotada pelas empresas com o objetivo de aumentar o controle, a acuracidade e a eficiência do estoque. Sistemas de gestão empresarial, como ERPs, permitem centralizar e padronizar os dados relacionados às operações, conectando departamentos como compras, vendas e logística, o que resulta em atualizações em tempo real e redução de erros humanos \cite{cast4it2024}.

Os softwares de gestão integrada consolidam dados operacionais e financeiros em uma única plataforma, permitindo que decisões estratégicas sejam tomadas com base em informações confiáveis e atualizadas. A automação dos processos, aliada à padronização e integração promovidas pelos sistemas ERP, contribui para a redução de desperdícios, otimização de recursos e aumento da produtividade das equipes. Dessa forma, empresas que aplicam essas soluções conseguem alinhar o gerenciamento de estoque às necessidades reais do mercado, ajustando os níveis de produtos de maneira mais eficiente e estratégica \cite{silva2020}.

Dessa forma, a gestão de estoque moderna deixa de ser apenas uma função operacional e passa a desempenhar um papel estratégico dentro das organizações. A adoção de tecnologias avançadas permite não apenas maior precisão e controle, mas também fortalece a capacidade competitiva das empresas, tornando o gerenciamento de estoque um elemento essencial para o sucesso e a sustentabilidade dos negócios.


\chapter{Gestão do Projeto}

Este capítulo apresenta como a equipe foi estruturada, quais papéis e responsabilidades cada membro assumiu, além da metodologia de gerenciamento adotada ao longo do desenvolvimento. Também são descritas as ferramentas utilizadas para organização das tarefas e o repositório da aplicação.

\section{Organização da Equipe}

A equipe do projeto foi composta por membros com múltiplas atribuições, garantindo flexibilidade e colaboração entre as áreas de desenvolvimento, testes e gestão. Cada integrante desempenhou funções combinadas de análise, desenvolvimento e suporte, de modo a cobrir todas as etapas do projeto sem sobrecarregar nenhum membro

Essa abordagem favoreceu a comunicação, a integração das atividades e a eficiência do trabalho, permitindo identificar problemas rapidamente e otimizar recursos. Além disso, a equipe teve papel ativo na orientação de funcionários da VIP PENHA quanto ao uso do sistema, garantindo que as novas práticas de gestão de estoque fossem incorporadas de forma eficaz.

\begin{table}[htbp] 
\centering
\begin{tabular}{|l|l|}
\hline
\textbf{Membro} & \textbf{Função(ões)} \\ \hline
Vitor           & Gestor, DBA         \\ \hline
Matheus         & Front-end, QA         \\ \hline
Beatriz         & Front-end, Back-end   \\ \hline
Rafael          & Front-end, Back-end         \\ \hline
Khalil              & Back-end, QA          \\ \hline
Gean            & QA                    \\ \hline
Marcelo         & QA         \\ \hline
\end{tabular}
\caption{Composição da equipe e funções}
\end{table}


\subsection{Responsabilidades/Papéis}

Os papéis desempenhados pela equipe foram definidos com base em suas competências e nas necessidades do projeto. A seguir, os principais papéis e suas responsabilidades:

\begin{itemize}
    \item \textbf{Gestor}: Responsável pelo planejamento, organização, acompanhamento das atividades, definição de cronogramas e mediação da comunicação interna e externa.
    \item \textbf{DBA (Administrador de Banco de Dados)}: Responsável pela modelagem do banco de dados, criação e manutenção das estruturas de dados, performance e integridade das informações.
    \item \textbf{Desenvolvedor Back-end}: Responsável pela implementação da lógica de negócio, criação de APIs, segurança e integração com o banco de dados.
    \item \textbf{Desenvolvedor Front-end}: Responsável pela criação da interface visual da aplicação, usabilidade, responsividade e interação com o usuário.
    \item \textbf{QA (Quality Assurance)}: Responsável pela garantia da qualidade da aplicação, planejamento e execução de testes funcionais, validação dos requisitos e registro de bugs.
\end{itemize}

\section{Metodologia de Gestão}

\subsection{Kanban}

A equipe optou pela utilização do método \textbf{Kanban} para o gerenciamento das atividades do projeto. Essa metodologia permite o acompanhamento visual focando na entrega contínua e em tempo real das tarefas.

O quadro Kanban utilizado possui as seguintes colunas:

\begin{itemize}
    \item \textbf{Backlog – Código:} Armazena ideias e funcionalidades relacionadas à implementação do código que ainda não foram iniciadas.
    \item \textbf{Backlog – Documentação:} Armazena tarefas de documentação que ainda não foram iniciadas.
    \item \textbf{Design:} Etapa dedicada à elaboração de pesquisas.
    \item \textbf{A Fazer:} Tarefas já priorizadas e planejadas, aguardando início.
    \item \textbf{Em Andamento:} Tarefas em desenvolvimento.
    \item \textbf{Revisão de Código:} Etapa de verificação e revisão do código antes da finalização.
    \item \textbf{Fase de Teste:} Validação e testes das funcionalidades desenvolvidas.
    \item \textbf{Concluído:} Tarefas finalizadas, revisadas e testadas com sucesso.
\end{itemize}

As atividades são constantemente avaliadas e realocadas entre as colunas conforme seu progresso, promovendo transparência e melhoria contínua do fluxo de trabalho.

\subsection{Funções da Equipe no Kanban}

A seguir, serão apresentadas as funções da equipe no projeto, com os respectivos membros atribuídos e as responsabilidades de cada papel:

\begin{table}[htbp]
\centering
\begin{tabular}{|l|l|l|}
\hline
\textbf{PO} & \textbf{Flow manager} & \textbf{Team Member} \\ \hline
Vitor        & Beatriz      & Matheus             \\ \hline
             &              & Khalil         \\ \hline
             &              & Gean                \\ \hline
             &              & Marcelo             \\ \hline
             &              & Rafael              \\ \hline
\end{tabular}
\caption{Atribuição de Papéis da Equipe no Kanban}
\end{table}

\FloatBarrier

\subsection{Responsabilidades de cada papel}

\begin{itemize}
    \item \textbf{PO (Product Owner)}: 
    Responsável por representar os interesses do cliente e das partes interessadas. No contexto Kanban, o PO prioriza as tarefas no backlog e garante que o trabalho mais valioso seja entregue primeiro, alinhando as entregas com os objetivos do projeto.
    
    \item \textbf{Flow Manager}: 
    Responsável por acompanhar e otimizar o fluxo de trabalho da equipe. Atua identificando gargalos, promovendo a melhoria contínua, monitorando métricas como lead time e WIP, além de incentivar a transparência, colaboração e boas práticas no uso do Kanban.
    
    \item \textbf{Team Members}: 
    São os membros da equipe de desenvolvimento que executam o trabalho técnico. Isso inclui análise, implementação, testes e revisão de código. Eles colaboram continuamente para manter o fluxo de trabalho saudável e entregar valor com qualidade.
\end{itemize}


\section{Repositório da Aplicação}

Nesta seção, definimos o repositório da aplicação, assim como seus links e necessidades para acesso.

\subsection{Definição do Repositório}

O repositório escolhido foi o Github, usado para armazenar, versionar e compartilhar o código-fonte do projeto. Assim, facilitando a colaboração entre os membros da equipe. Este repositório é público, o que significa que qualquer pessoa com o link pode visualizá-lo, navegar pelo código-fonte e acompanhar o histórico de alterações sem a necessidade de autenticação.

\subsection{Link e Acessos}

O repositório está hospedado no GitHub e pode ser acessado pelo seguinte link:

\begin{figure}[h!]
    \centering
    \includegraphics[width=0.3\textwidth]{Figuras/QR-CODE-GitHub.png}
    \caption{QR Code para o repositório no GitHub}
\end{figure}

\begin{center}
    \href{https://github.com/VitorDaSilvaOliveira/Projeto-Integrado-IFSP}{https://github.com/VitorDaSilvaOliveira/Projeto-Integrado-IFSP}
\end{center}

% ---
% primeiro capitulo de Resultados
% ---
\chapter{Desenvolvimento do Projeto}

Esta seção detalha o desenvolvimento do projeto, abrangendo desde a definição do escopo, regras de negócio e requisitos até as tecnologias utilizadas e a arquitetura adotada. O objetivo é descrever de forma clara como o sistema foi concebido, implementado e validado.
% ---
Fases do desenvolvimento do projeto
% ---
\section{Escopo do projeto}

O projeto tem como objetivo o desenvolvimento de um sistema de controle de estoque voltado para um estabelecimento comercial. O sistema será acessado via navegador e terá como foco a organização e o gerenciamento de produtos, fornecedores e movimentações de entrada e saída de estoque.


% ---


\subsection{Regras do Negócio}



\begin{quadro}[htb]
\centering
\caption{\label{quadro_rn1}Regras de Negócio (RN01 a RN04)}
\centering
\begin{tabular}{|p{1.6cm}|p{4.0cm}|p{7.5cm}|p{2.0cm}|}
    \hline
    \textbf{Código} & \textbf{Nome} & \textbf{Descrição} & \textbf{Requisito Relacionado} \\ \hline

    RN01 & Cadastro de Produtos & Produtos devem ter: nome, codigo, categoria, preço, garantia e estoque mínimo obrigatoriamente. & RF01 \\ \hline

RN02 & Atualização de Estoque & Qualquer entrada/saída deve atualizar automaticamente o estoque e recalcular o valor total em estoque. Produtos abaixo do estoque mínimo devem gerar notificações no sistema. & RF01, RF05 \\ \hline


RN03 & Registro de Movimentações & Toda movimentação deve gerar um registro contendo: tipo, data, responsável, produto(s), quantidade. & RF04 \\ \hline


RN04 & Controle de Acesso & O sistema deve ter diferentes níveis de acesso para perfis de Administrador e outros tipos de perfil. & RF07, RF08, RF11 \\ \hline

    

\end{tabular}
\end{quadro}

\FloatBarrier


\begin{quadro}[htb]
\centering
\caption{\label{quadro_rn2}Regras de Negócio (RN05 a RN09)}
\hspace*{-1cm}
\begin{tabular}{|p{1.6cm}|p{4.0cm}|p{7.5cm}|p{2.0cm}|}
    \hline
    \textbf{Código} & \textbf{Nome} & \textbf{Descrição} & \textbf{Requisito Relacionado} \\ \hline

 RN05 & Garantia & Devoluções só podem ser aceitas caso os itens ainda estejam dentro do prazo de garantia correspondente. & RF02, RF06 \\ \hline


RN06 & Validação de Venda & Vendas só podem ser registradas se houver estoque suficiente para todos os itens. & RF01, RF02 \\ \hline

RN07 & Informações da Venda & Toda venda deve ter informações relevantes como valor, desconto, forma de pagamento e dados do cliente, registrada nos dados do pedido, e na nota fiscal eletrônica. & RF02, RF12, RF13 \\ \hline

RN08 & Auditoria & Todas as exclusões e alterações de preço/estoque devem registrar IP, usuário, data/hora e valores antes/depois. & RF07 \\ \hline

RN09 & Criação de Relatórios & O sistema deve gerar relatórios com informações relevantes sobre produtos, pedidos, clientes, correspondendo ao estado atual do estoque, e esses relatórios devem poder ser exportados em formato adequado. & RF05 \\ \hline

\end{tabular}
\end{quadro}

\FloatBarrier




\subsection{Requisitos Funcionais}

\begin{quadro}[htb]
\centering
\caption{\label{quadro_rf1}Requisitos Funcionais (RF01 a RF02)}
\hspace*{-1cm}
\begin{tabular}{|p{1.4cm}|p{2.8cm}|p{4.5cm}|p{7.0cm}|}
    \hline
    \textbf{Código} & \textbf{Atores} & \textbf{Nome} & \textbf{Descrição} \\ \hline

  RF01 & Gerente de estoque & Controle de Produtos & O sistema deve permitir ao usuário cadastrar, alterar, consultar e deletar produtos. Deve registrar movimentações de estoque e o histórico de alterações, garantindo que produtos tenham categoria e quantidade mínima ideal. \\ \hline

  RF02 & Gerente de estoque, Vendedor/Funcionário & Controle de Pedidos & O sistema deve permitir cadastrar, consultar e alterar pedidos, vinculando produtos e quantidades. Pedidos aceitos geram movimentações de estoque, reduzindo saldos. Pedidos podem ter alteração de estado, mas não podem ser deletados. \\ \hline

   

   

    


\end{tabular}
\end{quadro}




\begin{quadro}[htb]
\centering
\caption{\label{quadro_rf2}Requisitos Funcionais (RF3 a RF10)}
\begin{tabular}{|p{1.4cm}|p{2.8cm}|p{4.5cm}|p{7.0cm}|}
    \hline
    \textbf{Código} & \textbf{Atores} & \textbf{Nome} & \textbf{Descrição} \\ \hline

       RF03 & Gerente de estoque & Controle de Fornecedores & O sistema deve permitir o cadastro de fornecedores. Após criado, o registro não pode ser deletado, apenas inativado, e deve ser relacionado a devoluções e movimentações. \\ \hline

    RF04 & Gerente de estoque & Controle de Movimentação & O sistema deve gerar movimentações a partir de pedidos e devoluções, alterando a quantidade em estoque. O registro é de segurança e não pode ser deletado após a criação. \\ \hline

    RF05 & Gerente de estoque & Relatórios de Estoque & O sistema deve gerar relatórios consolidados para auxiliar a tomada de decisão, identificando dados interessantes referentes a produtos, pedidos, clientes, fornecedores, movimentações e os relacionando de forma útil para a análise do negócio. \\ \hline

    RF06 & Vendedor, Gerente de estoque & Registrar Devoluções & O sistema deve permitir o registro de devoluções de produtos (de cliente ou para fornecedor), vinculadas a pedidos e dentro da garantia (para clientes). Deve registrar data, origem, justificativa e impactar relatórios e movimentações. \\ \hline

    RF07 & Administrador do sistema & Gestão de Usuários e Permissões & O sistema deve ter gestão de perfis (Admin/Gerente e Vendedor/Funcionário) com permissões distintas. Ações críticas (como deletar produtos) devem ser restritas ao Administrador e registradas em log. \\ \hline

    RF08 & Administrador do sistema, Usuário do sistema & Cadastro e Autenticação & O sistema deve permitir o cadastro e login com e-mail e senha, aplicando controle de acesso. As senhas devem ser armazenadas de forma segura (criptografia/hashing). \\ \hline

    RF09 & Usuário do sistema & Redefinição de Senha & O sistema deve permitir que usuários solicitem a redefinição de senha via e-mail. Deve enviar um link seguro e temporário, exigindo critérios mínimos de segurança para a nova senha. \\ \hline

    RF10 & Todos os usuários do sistema & Internacionalização & O sistema deve oferecer suporte a internacionalização, estando disponível em Português e Inglês. O idioma deve ser configurável pelo usuário ou ajustado pelo navegador. \\ \hline

    

    

    

\end{tabular}
\end{quadro}

\FloatBarrier

\begin{quadro}[htb]
\centering
\caption{\label{quadro_rf3}Requisitos Funcionais (RF11 a RF13)}
\hspace*{-1cm}
\begin{tabular}{|p{1.4cm}|p{2.8cm}|p{4.5cm}|p{7.0cm}|}
    \hline
    \textbf{Código} & \textbf{Atores} & \textbf{Nome} & \textbf{Descrição} \\ \hline

   RF11 & Administrador, Vendedor & Suporte a Múltiplas Lojas & O sistema deve permitir a gestão de estoque em múltiplas lojas. Usuários devem ser vinculados a uma loja, e o Administrador deve poder alternar entre elas. \\ \hline

    RF12 & Administrador, Vendedor & Controle de Clientes & O sistema deve permitir o gerenciamento e registro completo de clientes. Todo cliente registrado deve ter pedidos atrelados a ele. \\ \hline

 RF13 & Administrador, Vendedor & Geração de Nota Fiscal & O sistema deve, após a finalização de um pedido, gerar a Nota Fiscal eletrônica correspondente à venda, validando os dados fiscais do cliente e do produto, e registrando o status da emissão.. \\ \hline

  

\end{tabular}
\end{quadro}

\FloatBarrier


\subsection{Requisitos Não Funcionais}



\begin{quadro}[htb]
\centering
\caption{\label{quadro_rnf1}Requisitos Não Funcionais (RNF01 a RNF05)}
\hspace*{-1cm}
\begin{tabular}{|p{2.2cm}|p{4.0cm}|p{10.0cm}|}
    \hline
    \textbf{Código} & \textbf{Módulo} & \textbf{Descrição} \\ \hline
    RNF01 & Desempenho & O sistema deve suportar no mínimo 10 usuários simultâneos sem queda de desempenho. \\ \hline
    RNF02 & Segurança & O sistema deve ter todas as suas rotas protegidas por um mecanismo de autenticação robusto. \\ \hline
    RNF03 & Disponibilidade & O sistema deve estar completamente disponível a todo momento, com uma tolerância máxima de queda de 0,15 ao mês. \\ \hline
    RNF04 & Multiusuário & O sistema deve ser operável por múltiplos usuários ao mesmo tempo sem que ocorram inconsistências nos dados transacionais. \\ \hline
    RNF05 & Qualidade de Código & O sistema deve poder receber manutenção de código com facilidade, seguindo padrões de desenvolvimento e possuindo documentação extensiva. \\ \hline
RNF06 & Usabilidade & O sistema deve ter uma interface intuitiva e uma curva de aprendizagem rápida, permitindo que um novo usuário realize as principais operações com pouco treinamento. \\ \hline
RNF07 & Tolerância a Falhas & O sistema deve possuir um mecanismo de registro (logs) de erros e falhas, que permita a rastreabilidade e identificação da causa raiz em um tempo curto. \\ \hline
\end{tabular}
\end{quadro}

\FloatBarrier



\newpage
\section{Histórias de Usuário}

\begin{quadro}[htb]
\centering
\caption{Histórias de Usuário (US01 a US13)}
\label{hist_usuarios}
\begin{tabular}{|c|p{10cm}|p{4cm}|}
\hline
\textbf{Código} & \textbf{História de Usuário} & \textbf{Requisito Funcional Relacionado} \\
\hline
US01 & Cadastrar, Consultar, Alterar e Deletar Produtos. & \textbf{RF01} \\ \hline
US02 & Cadastrar e consultar pedidos. & \textbf{RF02} \\ \hline
US03 & Gerenciar fornecedores. & \textbf{RF03} \\ \hline
US04 & Registrar movimentações de estoque. & \textbf{RF04} \\ \hline
US05 & Gerar relatórios consolidados de estoque. & \textbf{RF05} \\ \hline
US06 & Registrar devoluções. & \textbf{RF06} \\ \hline
US07 & Gerenciar usuários e permissões. & \textbf{RF07} \\ \hline
US08 & Cadastro e autenticação segura. & \textbf{RF08} \\ \hline
US09 & Redefinição de senha segura. & \textbf{RF09} \\ \hline
US10 & Configurar o idioma do sistema. & \textbf{RF10} \\ \hline
US11 & Suporte a múltiplas lojas. & \textbf{RF11} \\ \hline
US12 & Gerenciar e registrar clientes. & \textbf{RF12} \\ \hline
US13 & Gerar a Nota Fiscal eletrônica. & \textbf{RF13} \\ \hline
\hline
\end{tabular}
\legend{Fonte: Elaborado com base nos Requisitos Funcionais (RF01 a RF12)}
\end{quadro}

\FloatBarrier

\subsection{Descrição das Histórias}

Neste tópico será detalhada a descrição das Histórias de Usuário, incluindo o ator principal, a descrição e os critérios de aceitação.

\begin{enumerate}

\item \textbf{US01: Cadastrar, Consultar, Alterar e Deletar Produtos}

\textbf{Descri\c{c}\~ao:} Como um \textbf{Gerente de Estoque}, eu quero ter a capacidade de gerenciar o ciclo de vida dos produtos no sistema (cadastrar, consultar, alterar e deletar), para que o catálogo de estoque esteja sempre atualizado e com a configuração da quantidade mínima ideal.

\textbf{Requisito Funcional Relacionado:} RF01 -- Controle de Produtos

\textbf{Crit\'erios de Aceita\c{c}\~ao:}
\begin{itemize}
  \item Deve existir uma interface acess\'ivel para o gerenciamento de produtos.
  \item O sistema deve permitir \textbf{registrar movimentações de estoque} e o \textbf{histórico de alterações} do produto.
  \item Deve ser obrigatório vincular o produto a uma \textbf{categoria} e definir a \textbf{quantidade mínima ideal}.
  \item O processo de exclusão deve garantir que o produto não tenha vínculos ativos.
\end{itemize}

\item \textbf{US02: Cadastrar e Consultar Pedidos}

\textbf{Descri\c{c}\~ao:} Como \textbf{Vendedor}, eu quero cadastrar, consultar e alterar pedidos, vinculando produtos e quantidades, para iniciar o processo de venda e garantir que o estoque seja reduzido após a aceitação do pedido.

\textbf{Requisito Funcional Relacionado:} RF02 -- Controle de Pedidos

\textbf{Crit\'erios de Aceita\c{c}\~ao:}
\begin{itemize}
  \item O sistema deve permitir vincular o pedido a um cliente (RF12) e a múltiplos produtos com suas quantidades.
  \item A \textbf{aceitação de um pedido} deve gerar automaticamente uma \textbf{movimentação de saída} no estoque (RF04).
  \item Pedidos podem ser \textbf{aceitos ou recusados}, mas \textbf{não devem ser deletados} para manter a rastreabilidade.
\end{itemize}

\item \textbf{US03: Gerenciar Fornecedores}

\textbf{Descri\c{c}\~ao:} Como \textbf{Gerente de Estoque}, eu quero cadastrar fornecedores com suas informações (nome, documento, localização), permitindo apenas a inativação, para que o histórico de devoluções e movimentações seja sempre rastreável.

\textbf{Requisito Funcional Relacionado:} RF03 -- Controle de Fornecedores

\textbf{Crit\'erios de Aceita\c{c}\~ao:}
\begin{itemize}
  \item O sistema deve permitir o cadastro de dados como CNPJ/Documento e Contatos.
  \item O fornecedor deve ser \textbf{relacionado a devoluções e movimentações} de entrada de produtos.
  \item Após a criação, o registro do fornecedor \textbf{não pode ser deletado}, apenas \textbf{inativado}.
\end{itemize}

\item \textbf{US04: Registrar Movimentações de Estoque}

\textbf{Descri\c{c}\~ao:} Como \textbf{Gerente de Estoque}, eu quero registrar movimentações de entrada ou saída (produto, quantidade, tipo, usuário), para que a quantidade em estoque seja atualizada de forma precisa.

\textbf{Requisito Funcional Relacionado:} RF04 -- Controle de Movimentação

\textbf{Crit\'erios de Aceita\c{c}\~ao:}
\begin{itemize}
  \item O sistema deve permitir registrar o \textbf{tipo de movimentação} (Ex: Entrada por Compra, Saída por Venda, Perda).
  \item A quantidade em estoque deve ser alterada automaticamente após o registro.
  \item O registro de movimentação é de \textbf{segurança} e \textbf{não pode ser deletado} após a criação.
\end{itemize}

\item \textbf{US05: Gerar Relatórios Consolidados de Estoque}

\textbf{Descri\c{c}\~ao:} Como \textbf{Gerente de Estoque}, eu quero gerar relatórios consolidados, com filtros de loja, para identificar produtos abaixo do mínimo ideal, perdas e analisar o saldo total considerando todas as movimentações.

\textbf{Requisito Funcional Relacionado:} RF05 -- Relatórios de Estoque

\textbf{Crit\'erios de Aceita\c{c}\~ao:}
\begin{itemize}
  \item O relatório deve ser capaz de \textbf{identificar produtos abaixo do mínimo ideal} (RF01).
  \item Deve incluir informações sobre perdas, entradas, saídas e devoluções.
  \item O relatório deve permitir filtro por \textbf{loja} (RF11).
\end{itemize}

\item \textbf{US06: Registrar Devoluções}

\textbf{Descri\c{c}\~ao:} Como \textbf{Vendedor ou Gerente de Estoque}, eu quero registrar devoluções (de cliente ou para fornecedor), vinculadas a pedidos, para que o estoque e os relatórios sejam corretamente impactados.

\textbf{Requisito Funcional Relacionado:} RF06 -- Registrar Devoluções

\textbf{Crit\'erios de Aceita\c{c}\~ao:}
\begin{itemize}
  \item Deve ser obrigatório registrar a \textbf{origem}, \textbf{justificativa} e a \textbf{data da devolução}.
  \item Devoluções de cliente devem ser validadas se estão \textbf{dentro do período de garantia}.
  \item O registro deve gerar uma \textbf{movimentação de entrada} no estoque (RF04) ou de saída para o fornecedor.
\end{itemize}

\item \textbf{US07: Gerenciar Usuários e Permissões}

\textbf{Descri\c{c}\~ao:} Como \textbf{Administrador do Sistema}, eu quero ter gestão de perfis (Admin/Gerente e Vendedor/Funcionário) com permissões distintas, para que ações críticas sejam restritas e registradas em log.

\textbf{Requisito Funcional Relacionado:} RF07 -- Gestão de Usuários e Permissões

\textbf{Crit\'erios de Aceita\c{c}\~ao:}
\begin{itemize}
  \item O sistema deve garantir que ações críticas (ex: exclusão de registros) sejam \textbf{restritas ao perfil Administrador}.
  \item Todas as alterações de permissão devem ser \textbf{registradas em log} de auditoria.
  \item A interface deve ser ajustada para exibir apenas as funcionalidades permitidas ao perfil logado.
\end{itemize}

\item \textbf{US08: Cadastro e Autenticação Segura}

\textbf{Descri\c{c}\~ao:} Como \textbf{Usuário do Sistema}, eu quero poder me cadastrar e efetuar o login usando e-mail e senha, para ter acesso controlado e seguro ao sistema.

\textbf{Requisito Funcional Relacionado:} RF08 -- Cadastro e Autenticação

\textbf{Crit\'erios de Aceita\c{c}\~ao:}
\begin{itemize}
  \item O sistema deve aplicar \textbf{controle de acesso} e redirecionar para a tela inicial após o login.
  \item As senhas devem ser armazenadas de forma segura (utilizando \textbf{criptografia/hashing}).
  \item Deve ser verificada a existência do usuário para evitar duplicidade.
\end{itemize}

\item \textbf{US09: Redefinição de Senha Segura}

\textbf{Descri\c{c}\~ao:} Como \textbf{Usuário do Sistema}, eu quero solicitar a redefinição de senha, recebendo um link seguro e temporário via e-mail, para recuperar meu acesso quando necessário.

\textbf{Requisito Funcional Relacionado:} RF09 -- Redefinição de Senha

\textbf{Crit\'erios de Aceita\c{c}\~ao:}
\begin{itemize}
  \item O link de redefinição enviado por e-mail deve ter \textbf{validade limitada} (temporário).
  \item O sistema deve exigir \textbf{critérios mínimos de segurança} (ex: tamanho, caracteres especiais) para a nova senha.
  \item Deve haver uma confirmação na tela após a solicitação e após a redefinição bem-sucedida.
\end{itemize}

\item \textbf{US10: Configurar o Idioma do Sistema}

\textbf{Descri\c{c}\~ao:} Como \textbf{Qualquer Usuário do Sistema}, eu quero que o sistema ofereça suporte a internacionalização, estando disponível em, no mínimo, Português e Inglês, para configurar a interface no meu idioma.

\textbf{Requisito Funcional Relacionado:} RF10 -- Internacionalização

\textbf{Crit\'erios de Aceita\c{c}\~ao:}
\begin{itemize}
  \item Deve haver uma opção para o usuário \textbf{configurar o idioma preferencial} no seu perfil.
  \item Alternativamente, o sistema deve \textbf{ajustar o idioma} com base na configuração do navegador.
  \item Todos os textos da interface de usuário devem estar traduzidos para os idiomas suportados.
\end{itemize}

\item \textbf{US11: Suporte a Múltiplas Lojas}

\textbf{Descri\c{c}\~ao:} Como \textbf{Administrador}, eu quero que o sistema permita a gestão de estoque em, no mínimo, duas lojas distintas, para que os usuários sejam vinculados a uma loja e eu possa alternar entre elas.

\textbf{Requisito Funcional Relacionado:} RF11 -- Suporte a Múltiplas Lojas

\textbf{Crit\'erios de Aceita\c{c}\~ao:}
\begin{itemize}
  \item Deve ser possível \textbf{vincular usuários a uma loja específica} (exceto o Administrador).
  \item O Administrador deve ter uma interface para \textbf{alternar o contexto} entre as lojas.
  \item Todas as consultas e relatórios de estoque (RF05) e movimentações (RF04) devem respeitar o contexto da loja ativa.
\end{itemize}

\item \textbf{US12: Gerenciar e Registrar Clientes}

\textbf{Descri\c{c}\~ao:} Como \textbf{Vendedor}, eu quero gerenciar o cadastro completo de clientes (nome, documento, contatos), para que todo pedido de venda esteja atrelado a um cliente registrado.

\textbf{Requisito Funcional Relacionado:} RF12 -- Controle de Clientes

\textbf{Crit\'erios de Aceita\c{c}\~ao:}
\begin{itemize}
  \item O sistema deve permitir o cadastro de dados como CPF/CNPJ, e-mail e telefone.
  \item O sistema \textbf{deve exigir um cliente registrado} para a criação de um novo pedido (RF02).
  \item Deve ser possível consultar o histórico de pedidos e devoluções (RF06) atrelado a cada cliente.
\end{itemize}

\item \textbf{US13: Gerar a Nota Fiscal Eletrônica}

\textbf{Descri\c{c}\~ao:} Como \textbf{Vendedor}, eu quero que o sistema \textbf{gere a Nota Fiscal eletrônica} automaticamente após a aceitação ou finalização de um pedido (RF02), para cumprir as obrigações fiscais e disponibilizar o documento ao cliente.

\textbf{Requisito Funcional Relacionado:} RF13 -- Geração de Nota Fiscal

\textbf{Crit\'erios de Aceita\c{c}\~ao:}
\begin{itemize}
  \item A geração da NF-e deve ser disparada após a \textbf{confirmação de que o pedido foi aceito e o pagamento efetuado}.
  \item O sistema deve \textbf{validar os dados fiscais} antes da emissão.
  \item Deve ser registrado o \textbf{status da emissão} e a chave de acesso da NF-e.
  \item O sistema deve \textbf{enviar a NF-e (XML e DANFE)} ao e-mail do cliente automaticamente.
\end{itemize}

\end{enumerate}


\section{Arquitetura}

A arquitetura do sistema representa a estrutura organizacional fundamental da aplicação, incluindo seus principais componentes, suas relações, ambientes de execução e os padrões adotados para garantir a escalabilidade, manutenibilidade e segurança. Esta seção descreve a forma como o sistema foi planejado e dividido, visando facilitar tanto o desenvolvimento quanto a futura evolução da solução. São considerados aqui os aspectos lógicos e físicos do sistema, incluindo frameworks, tecnologias e infraestrutura.


\subsection{Definições da Arquitetura}

A arquitetura do projeto foi desenvolvida como uma aplicação Web, apoiando-se em tecnologias modernas para garantir robustez, escalabilidade e segurança. O sistema foi concebido com uma abordagem em camadas, utilizando principalmente tecnologias da plataforma .NET.

No lado do servidor, foi adotada a linguagem C\# com o framework ASP.NET MVC, facilitando a separação das responsabilidades e promovendo uma melhor organização do código através do padrão Model-View-Controller (MVC).

Para o acesso a dados, foi utilizado Entity Framework com LINQ, que permite uma comunicação eficiente com o banco de dados relacional e facilita a execução de consultas através de expressões C\#. A base de dados principal é gerida por Microsoft SQL Server, garantindo a persistência e segurança das informações.

No front-end, a aplicação utiliza Razor Views (.cshtml), combinadas com HTML, CSS e JavaScript, para renderização dinâmica das páginas diretamente no servidor. O uso de Bootstrap assegura uma interface responsiva e compatível com diferentes dispositivos. Para manipulação do DOM e criação de interações dinâmicas, foi utlizado o jQuery, ampliando a usabilidade da aplicação.

Para aprimorar a interação do usuário, foi utilizada a biblioteca jQuery, que facilita o tratamento de eventos e a manipulação do DOM, permitindo a criação de funcionalidades dinâmicas e responsivas.

A autenticação e autorização dos usuários são tratadas por meio do ASP.NET Identity, que integra recursos robustos de controle de acesso, gerenciamento de permissões e segurança na autenticação. Esse serviço está destacado no diagrama como um componente separado e essencial para a integridade e confiabilidade do sistema.

Além disso, o sistema utiliza o framework JJMasterData, uma solução que permite a criação dinâmica de formulários e estruturas de dados em tempo de execução. Com ele, é possível configurar cadastros e regras de negócio diretamente pela interface, sem necessidade de mudanças no código, trazendo mais flexibilidade e agilidade na evolução do sistema.

\subsection{Diagramas da Arquitetura}

Foram construídos diagramas de Componentes e de Implantação para visualizar a estrutura do projeto. O diagrama de componentes descreve a estrutura modular da aplicação, evidenciando os elementos lógicos e suas interações internas. Já o diagrama de implantação representa a distribuição física desses componentes em ambientes de execução, detalhando a infraestrutura utilizada e as conexões entre os ambientes.

\begin{figure}[h!]
\hspace*{+0,2cm}
    \centering
\includegraphics[width=1.0\textwidth]{Figuras/Componentes.png}
    \caption{Diagrama de Componentes}
\end{figure}

\begin{figure}[h!]
\hspace*{-1,2cm}
    \centering
\includegraphics[width=1.0\textwidth]{Figuras/Implantação.png}
    \caption{Diagrama de Implantação}
\end{figure}



\FloatBarrier


\section{Tecnologias}

Esta seção detalha as tecnologias escolhidas para o desenvolvimento do sistema, organizadas em categorias. A seleção foi baseada em critérios de desempenho, escalabilidade e compatibilidade com os requisitos do projeto.

\subsection{Front-end}

\begin{itemize}

   \item \textbf{Razor Views}: Arquivos .cshtml que utilizam a engine Razor do ASP.NET para gerar páginas HTML dinâmicas. Essa abordagem permite mesclar código C\# com marcação HTML de forma fluida, facilitando a renderização de conteúdo no lado do servidor e a reutilização de componentes visuais \cite{Razor}.

   \item \textbf{HTML, CSS e JavaScript}: Tecnologias fundamentais para a construção da camada de apresentação do sistema. O HTML estrutura o conteúdo das páginas, o CSS define o estilo visual e o JavaScript adiciona interatividade ao front-end \cite{FrontEnd}.

   \item \textbf{Bootstrap}: Framework front-end baseado em HTML, CSS e JS que oferece componentes prontos e responsivos, facilitando a criação de interfaces modernas, padronizadas e adaptáveis a diferentes dispositivos \cite{Bootstrap}.

   \item \textbf{jQuery}: Biblioteca JavaScript que simplifica a manipulação do DOM, o tratamento de eventos e requisições AJAX. Foi utilizada para agilizar o desenvolvimento de funcionalidades interativas no front-end da aplicação \cite{JQuery}.

\end{itemize}


\subsection{Back-end}

\begin{itemize}

  \item \textbf{C\# com ASP.NET MVC}: Linguagem e framework utilizados na construção do back-end da aplicação, seguindo o padrão arquitetural Model-View-Controller (MVC), que organiza o código de forma modular, separando lógica de negócios, visualização e controle \cite{AspNet2025}.

  \item \textbf{Entity Framework com LINQ}: Conjunto de tecnologias para acesso a dados em .NET. O Entity Framework permite o mapeamento objeto-relacional (ORM), e o LINQ facilita a realização de consultas ao banco de dados de forma legível e integrada ao C\# \cite{EfCore}.

  \item \textbf{ASP.NET Identity}: Sistema de autenticação e controle de acesso da Microsoft utilizado para gerenciar usuários, perfis e permissões de forma segura e integrada ao projeto \cite{Identity}.

  \item \textbf{JJMasterData}: Ferramenta de administração e modelagem de dados que permite a criação dinâmica de formulários e telas de cadastro com base nas entidades do sistema, otimizando o desenvolvimento da interface administrativa \cite{JJMasterdata}.

\end{itemize}

\subsection{Banco de Dados}

\begin{itemize}
    \item \textbf{Microsoft SQL Server}: O Microsoft SQL Server é um sistema de gerenciamento de banco de dados relacional (RDBMS) robusto e amplamente utilizado no mercado, responsável pelo armazenamento seguro das informações da aplicação \cite{SqlServer2025}.


\end{itemize}

\subsection{Infraestrutura}

\begin{itemize}
    \item \textbf{Microsoft Azure}: Plataforma de computação em nuvem utilizada para hospedar a aplicação e seus serviços relacionados. A utilização do Azure proporciona escalabilidade, segurança e alta disponibilidade \cite{Azure2025}. O Azure está sendo utilizado como plataforma principal para a hospedagem de todos os componentes do sistema, abrangendo o banco de dados, o back-end e o front-end.
\end{itemize}

\section{Estilos de Codificação e Padrões de Projeto}
Para manter a consistência e legibilidade do código-fonte, foram adotados os seguintes padrões:

\begin{itemize}
    \item \textbf{C\# (Back-end) - Padrões .NET:} Utilização das diretrizes oficiais do Microsoft .NET, que incluem:
    \begin{itemize}
        \item \textbf{Convenções de Nomenclatura:} Aplicação de \textit{PascalCase} para nomes de classes, métodos, propriedades (e.g., \texttt{ProdutoController}) e \textit{camelCase} para variáveis locais (e.g., \texttt{faturamentoAcumulado}).
        \item \textbf{Nullability:} Uso mandatório de \textit{nullable reference types} (\texttt{string?}, \texttt{DateTime?}) para mitigar erros de referência nula. O tipo \texttt{decimal?} foi utilizado para campos opcionais de valores monetários.
        \item \textbf{Tipagem e Performance:} Uso de construtores primários em Services (e.g., \texttt{PedidosService(EstoqueDbContext context)}) e utilização de \textit{Records} ou \textit{Value Tuples} para retorno eficiente de múltiplos valores (e.g., \texttt{Task<(List<string> labels, List<int> vendas, List<int> trocas)>}).
    \end{itemize}

    \item \textbf{JavaScript/Front-end:} Aplicação de linter (\textit{ESLint}) para impor padrões como \textit{camelCase} e garantir a sintaxe moderna e consistente. Foi adotado o padrão de injetar variáveis de tradução do C\# (\texttt{@Localizer["Chave"]}) em um objeto JS (\texttt{textTranslated}) para garantir a localização do conteúdo dinâmico do \textit{Chart.js}.

    \item \textbf{Arquitetura e Convenções MVC (ASP.NET Core):} Separação estrita de responsabilidades, conforme as convenções:
    \begin{itemize}
        \item \textbf{Controle e Roteamento:} Utilização de \textit{Areas} (\texttt{/Areas/Estoque/}) para organizar módulos grandes e \textit{Controllers} para roteamento e orquestração.
        \item \textbf{Lógica de Negócio (Services):} O \textit{Controller} delega a lógica de negócio aos \textit{Services} (Injeção de Dependência). Foi imposto que toda consulta complexa do Entity Framework Core (EF Core), como \texttt{GroupBy} e \texttt{Sum}, seja encapsulada dentro do \texttt{Service}.
 
    \end{itemize}
\end{itemize}

\section{Cobertura de Código e Testes} 
A Cobertura de Código é uma métrica essencial que estabelece o nível de \textbf{confiabilidade e robustez} do código-fonte. E a definição estratégica de teste é fundamental para garantir essa qualidade no codigo e no funcionamento do sistema como um todo.

\subsection{Metas e Escopo da Cobertura}
A estratégia de cobertura e qualidade foi delineada para garantir a máxima confiabilidade nas áreas de maior risco do sistema. O escopo foi rigorosamente focado em três pilares:

\begin{itemize}
    \item \textbf{Cobertura de Classes:} Foi estabelecida uma meta mínima de 80\% de cobertura de classes  para todos os módulos que encapsulam a lógica de negócio e complexidade algorítmica.
    \item \textbf{Segurança:} O ambiente web foi submetido a testes de segurança para garantir que o sistema esteja de acordo com os padrões de segurança, recebendo no mínimo uma nota A.
    \item \textbf{Integração:}Os testes de integração são essenciais para garantir que todos os ambientes necessários para que a aplicação funcione, estejam devidamente integrados, sem a existência de erros e falhas de segurança.
    \item \textbf{Usabilidade:} O esforço destes testes foi concentrado em fornecer a melhor experiência para o usuário, garantindo que todas as telas e processos sejam de fácil entedimento e interação.
\end{itemize}

\subsection{Implementação e Validação}
A qualidade do código e a aderência à meta de cobertura são asseguradas por uma combinação estratégica de testes manuais e automatizados, integrados ao fluxo de desenvolvimento.

\begin{itemize}
    \item \textbf{Testes Unitários e de Integração:  (C\#):}
 
        A cobertura de classes é implementada pelos \textit{Unit Tests} (C\#, que indicam a porcentagem de classes cobertas, e a integração dos ambientes é feita através de \textit{Integration Tests} (C\#), que indicam se a conexão entre os sitemas está funcional ou falha.

    \item \textbf{Testes de Ponta-a-Ponta (\textit{End-to-End}, E2E):}

        A validação dos fluxos completos de usuário, incluindo a interface gráfica, é realizada com \textbf{testes automatizados usando Selenium}, garantindo que o processo, sob a perspectiva do usuário, esteja funcional. E também foram feitos tetes manuais para validar a usabilidade do sistema.


    \item \textbf{Teste de Segurnça:}

        O ambiente web do projeto foi submetido a testes de segurança em duas ferramentas: SSL Labs e Security Headers, ferramentas que avaliam se o ambiente web está de acordo com os padrões de segurança mais modernos.
\end{itemize}



\section{Resultados dos Testes}

Após definidas as metas de cobertura e os métodos de implementação e validação, a seguir consta os resultados dos testes propostos.

\subsection{Cobertura de Classes}

O projeto da VIP Penha é formado por diversas classes com diferentes funções, como mapeamento de tabelas do banco de dados, classes controladoras de requisições e classes que executam as regras de negócio.
Foram Mapeadas 85 classes em nosso sistema, analisando o conteúdo e a relevância de cada classe, decidimos por testar 83,7\% delas, totalizando 71 classes. 
As classes escolhidas para a testagem foram:
\begin{itemize}
\item \textbf{As que envolviam entidades (classes mapeadoras de tabelas do banco de dados) que executam papéis de grande importância nos processos do negócio.}
\item \textbf{As classes de serviço que executam métodos das regras de negócio}
\item \textbf{As classes que contêm as rotas que aceitam requisições (controladoras)}
\item \textbf{As classes que geram as telas do sistema com base na biblioteca JJ Masterdata}
\end{itemize}

\begin{quadro}[htb]
\centering
\caption{Resumo Consolidado Estoque.Domain}
\label{res_domain}
\begin{tabular}{|c|c|p{4cm}|p{4cm}|}
\hline
\textbf{Tipo} & \textbf{Total} & \textbf{Candidatas a teste} & \textbf{Cobertura esperada} \\
\hline
Entities & 22 & 20 & 91\% \\ \hline
Enums & 7 & 4 & 57\% \\ \hline
Models & 13 & 9 & 69\% \\ \hline
Total Geral & 42 & 33 & 72\% \\ \hline

\hline
\end{tabular}
\legend{Fonte: Elaborado pelos autores}
\end{quadro}

\FloatBarrier

\begin{quadro}[htb]
\centering
\caption{Resumo Consolidado Estoque.Infrastructure}
\label{res_infrastructure}
\begin{tabular}{|c|c|p{4cm}|p{4cm}|}
\hline
\textbf{Tipo} & \textbf{Total} & \textbf{Candidatas a teste} & \textbf{Cobertura esperada} \\
\hline
Data & 1 & 1 & 100\% \\ \hline
Factory & 2 & 1 & 50\% \\ \hline
Services & 18 & 17 & 94\% \\ \hline
Total Geral & 26 & 24 & 86\% \\ \hline

\hline
\end{tabular}
\legend{Fonte: Elaborado pelos autores}
\end{quadro}

\FloatBarrier

\begin{quadro}[htb]
\centering
\caption{Resumo Consolidado Estoque.Web}
\label{res_web}
\begin{tabular}{|c|c|p{4cm}|p{4cm}|}
\hline
\textbf{Tipo} & \textbf{Total} & \textbf{Candidatas a teste} & \textbf{Cobertura esperada} \\
\hline
Admin (Controllers) & 6 & 6 & 100\% \\ \hline
Estoque (Controllers) & 10 & 10 & 100\% \\ \hline
Identity (Controllers) & 4 & 4 & 100\% \\ \hline
Configuration & 14 & 8 & 73\% \\ \hline
Total Geral & 34 & 28 & 93\% \\ \hline

\hline
\end{tabular}
\legend{Fonte: Elaborado pelos autores}
\end{quadro}

\FloatBarrier

Todas as classes estão mapeadas e disponíveis no arquivo do projeto em: \\
https://github.com/VitorDaSilvaOliveira/Projeto-Integrado-IFSP/tree/master/docs/Anexos.



\subsection{Segurança}

O ambiente web do projeto foi submetido a testes de segurança em duas ferramentas: SSL Labs e Security Headers obtivemos resultados positivos em ambos, o que demonstra a conformidade das práticas adotadas com os padrões modernos de segurança da web.

\begin{itemize}
 \item \textbf{Avaliação SSL Labs — Nota A}
O domínio do projeto recebeu nota A no teste do SSL Labs, o que indica uma implementação sólida e segura do protocolo HTTPS/TLS.
 Os critérios avaliados apresentaram os seguintes resultados:


\begin{figure}[h!]
    \centering
    \includegraphics[width=0.8\textwidth]{Figuras/SSL.png}
    \caption{Resultado do SSL Labs}
\end{figure}

\begin{quadro}[htb]
\centering
\caption{Resultado SSL Labs}
\label{res_ssl}
\begin{tabular}{|c|c|p{6cm}|}
\hline
\textbf{Critério} & \textbf{Pontuação} & \textbf{Explicação} \\
\hline
Certificate & 100/100 & O certificado SSL/TLS está corretamente configurado, é válido, confiável e emitido por uma autoridade certificadora reconhecida. Garante autenticidade e confiança na comunicação. \\ \hline
Protocol Support & 100/100 & Apenas versões seguras dos protocolos TLS estão habilitadas, com desativação de versões obsoletas (como SSLv3 e TLS 1.0/1.1), evitando vulnerabilidades conhecidas. \\ \hline
Key Exchange & 90/100 & O mecanismo de troca de chaves usa algoritmos modernos (como ECDHE) que fornecem sigilo perfeito (Perfect Forward Secrecy), garantindo que sessões passadas não possam ser descriptografadas mesmo em caso de comprometimento futuro da chave privada. \\ \hline
Cipher Strength & 90/100 & Os conjuntos de cifras (ciphers) utilizados são fortes, com chaves de 256 bits e algoritmos seguros (AES-GCM, SHA-256), protegendo os dados contra ataques de força bruta e descriptografia não autorizada. \\ \hline

\hline
\end{tabular}
\legend{Fonte: Elaborado pelos autores}
\end{quadro}

\FloatBarrier

 \item \textbf{Avaliação Security Headers}
O projeto também foi analisado pela ferramenta Security Headers, que verifica a presença e configuração de cabeçalhos HTTP responsáveis por mitigar ataques comuns (como XSS, clickjacking e data leaks).

\begin{figure}[h!]
    \centering
    \includegraphics[width=0.8\textwidth]{Figuras/Headers.png}
    \caption{Resultado do Security Headers}
\end{figure}

O site obteve todas as diretivas essenciais ativas, conforme o checklist abaixo:

\begin{quadro}[htb]
\centering
\caption{Resultado Security Headers}
\label{res_shd}
\begin{tabular}{|c|c|p{6cm}|}
\hline
\textbf{Cabeçalho} & \textbf{Status} & \textbf{Função} \\
\hline
Strict-Transport-Security (HSTS) & Positivo & Obriga o navegador a se conectar sempre via HTTPS, evitando downgrade attacks. \\ \hline
Permissions-Policy & Positivo & Restringe o acesso a APIs e recursos sensíveis (como câmera, microfone e geolocalização). \\ \hline
Content-Security-Policy (CSP) & Positivo & Impede a execução de scripts não autorizados, protegendo contra Cross-Site Scripting (XSS). \\ \hline
Referrer-Policy & Positivo & Controla o envio de informações de origem, reduzindo o vazamento de dados de navegação. \\ \hline
X-Content-Type-Options & Positivo & Evita que navegadores interpretem arquivos com tipos MIME incorretos, prevenindo ataques de injeção. \\ \hline
X-Frame-Options & Positivo & Bloqueia o carregamento do site em iframes externos, prevenindo ataques de clickjacking. \\ \hline

\hline
\end{tabular}
\legend{Fonte: Elaborado pelos autores}
\end{quadro}

\FloatBarrier


\end{itemize}

Conclusões a respeito de segurança
Os resultados das análises indicam que o site está altamente seguro, contendo boas práticas como
\begin{itemize}
 \item \textbf{Comunicação criptografada e confiável.}
 \item \textbf{Cabeçalhos HTTP de segurança corretamente configurados.}
 \item \textbf{Mitigação de ataques comuns de rede e navegador.}
\end{itemize}

\subsection{Integração}

A aplicação da VIP Penha necessita de duas integrações para funcionar devidamente, sendo elas a conexão com um banco de dados SQL Server e a integração com um ambiente de emissão de notas fiscais.

\begin{itemize}

 \item \textbf{Integração com o Banco de Dados}

O projeto realiza sua integração com o banco de dados \textit{SQL Server} por meio do \textit{Entity Framework Core} (EF Core), um ORM (Object-Relational Mapper) que permite mapear as entidades de domínio diretamente para tabelas do banco, garantindo abstração da camada de persistência e manutenção simplificada. A configuração dessa integração é feita na classe \textit{ConfigurationExtensions}, onde o contexto \textit{EstoqueDbContext} é registrado com a string de conexão \textit{DefaultConnection}
Essa abordagem permite que o projeto: Utilize injeção de dependência para gerenciar o contexto de dados, centralize a migração de esquemas dentro do assembly Estoque.Infrastructure e mantenha consistência entre as entidades de domínio (Estoque.Domain.Entities) e suas representações no banco. Além disso, o uso de repositórios e serviços garante que as operações de CRUD e as transações sejam tratadas de forma controlada, assegurando integridade e rastreabilidade dos dados.


 \item \textbf{Integração com o Ambiente da Nota Fiscal}

O ambiente da Nota Fiscal foi integrado ao sistema por meio do serviço  \textit{NotaFiscalService}, localizado na camada  \textit{Estoque.Infrastructure.Services}.
Esse serviço tem como responsabilidade principal o gerenciamento dos processos de emissão, registro e comunicação das notas fiscais, integrando o sistema a um ambiente fiscal externo (seja de homologação ou produção).
A integração segue princípios de arquitetura limpa, sendo mediada por:

\begin{itemize}
 \item \textbf{Serviços especializados responsáveis por comunicação HTTP segura com a API fiscal.}
 \item \textbf{Modelos de dados padronizados que representam os documentos fiscais eletrônicos.}
 \item \textbf{Tratamento de exceções e logs detalhados para auditoria e rastreabilidade.}
\end{itemize}

Essa integração é configurada e injetada automaticamente pelo método AddEstoqueServices(), que registra o NotaFiscalService e outros componentes auxiliares:
 \textit{services.AddScoped<NotaFiscalService>()};
Dessa forma, o ambiente de notas fiscais é completamente integrado ao fluxo do sistema, permitindo:

\begin{itemize}
 \item \textbf{Automatização da emissão de notas após o registro de vendas ou movimentações.}
 \item \textbf{Sincronização de status fiscais (autorização, cancelamento, inutilização)}
 \item \textbf{Segurança no tráfego de dados por meio de HTTPS e certificados digitais válidos.}
\end{itemize}

 \item \textbf{Conclusão sobre Integração}

A integração entre o sistema, o banco de dados e o ambiente da nota fiscal foi projetada de forma modular e segura. Enquanto o  \textit{Entity Framework Core} assegura persistência confiável e manutenção simples dos dados internos, o  \textit{NotaFiscalService} garante interoperabilidade com os sistemas fiscais externos, proporcionando automação, conformidade legal e rastreabilidade completa das operações.

\end{itemize}

\subsection{Usabilidade}

Testes de usabilidade do sistema foram executados almejando a melhor experiência possível para o usuário. Foram apresentadas as visões gerais do sistema e definidos objetivos específicos a serem alcançados durante a utilização por um usuário externo ao desenvolvimento do projeto. Inicialmente, recebemos feedbacks negativos quanto à clareza dos processos e à disposição de algumas funcionalidades na interface.

Com base nessas observações, foram revisados os fluxos de navegação e reorganizados elementos visuais para tornar o sistema mais intuitivo. O participante relatou dificuldades principalmente na compreensão da sequência correta de ações, o que indicou a necessidade de uma revisão nos fluxos dos processos e das interfaces gráficas.

Após as melhorias, uma nova rodada de testes foi realizada, o usuário conseguiu realizar as tarefas propostas com muito mais fluidez e clareza. Desta forma, foi constatada a eficácia das modificações, auxiliando na evolução do produto final.



\section{Segurança, Privacidade, Legislação}

Este tópico é dedicado a explicar sobre as questões de segurança e legislação relevantes para o projeto.

\subsection{Critérios de Segurança e Privacidade}

O sistema de gerenciamento de estoque foi projetado com critérios de segurança e privacidade para garantir a integridade das informações e proteger os dados dos usuários. As principais medidas adotadas no projeto incluem:

\begin{itemize}
    \item \textbf{Autenticação e Autorização:} O acesso ao sistema é controlado por meio de autenticação de usuários, utilizando o \textit{ASP.NET Identity}. Cada usuário precisa estar autenticado para acessar funcionalidades sensíveis, como cadastro de produtos, movimentação de estoque ou relatórios.

    \item \textbf{Criptografia de Senhas:} As senhas dos usuários são armazenadas de forma criptografada (hash) no banco de dados, conforme práticas recomendadas pelo \textit{Entity Framework}.

    \item \textbf{Boas Práticas de Privacidade:} Os dados pessoais dos usuários, como nome e e-mail, são utilizados apenas para fins de autenticação e gerenciamento interno, e não são compartilhados com terceiros.
\end{itemize}

\subsection{Legislação}

A  Lei Geral de Proteção de Dados (LGPD - Lei nº 13.709/2018) garante transparencia diante ao uso de dados pessoais garantindo que os dados coletados sejam utilizados exclusivamente para fins de autenticação, autorização e controle de acesso ao sistema. Sendo assim, o sistema segue as diretrizes da LGPD, adotando medidas técnicas e administrativas adequadas para proteger os dados pessoais contra acessos não autorizados, perda ou vazamento. 

Além da LGPD, o sistema também considera princípios estabelecidos no Marco Civil da Internet (Lei nº 12.965/2014), que estabelece garantias, direitos e deveres para o uso da internet no Brasil. Nesse sentido, o projeto reflete esses princípios ao promover a proteção da privacidade dos usuários, a segurança das informações e a transparência nas operações de tratamento de dados, assegurando que o uso da aplicação respeite os direitos fundamentais de liberdade e privacidade no ambiente digital.




\section{Modelo de Banco de Dados}

O modelo de banco de dados é responsável por organizar e estruturar as informações utilizadas no sistema, garantindo integridade, consistência e facilidade no acesso aos dados. O MER (Modelo Entidade-Relacionamento) e o DER (Diagrama Entidade-Relacionamento) representam graficamente as entidades, atributos e relacionamentos do sistema, enquanto o dicionário de dados descreve detalhadamente cada campo presente no banco, incluindo tipo, tamanho e função. O modelo foi projetado para gerenciar de forma eficiente o controle de estoque, permitindo registrar entradas, saídas, usuários, produtos, fornecedores e categorias.

\begin{figure}[h!]
    \centering
\includegraphics[width=1.0\textwidth]{Figuras/MERestoque.png}
    \caption{Modelo de Entidade-Relacionamento}
\end{figure}


\FloatBarrier


\begin{figure}[h!]
    \centering
\includegraphics[width=1.0\textwidth]{Figuras/DERestoque.png}
    \caption{Diagrama de Entidade-Relacionamento}
\end{figure}

\FloatBarrier







% coloca o dicionário de dados?  \subsection{Dicionário de Dados}

\section{Cronograma}

Esta seção apresenta uma visão detalhada das principais fases do projeto, focando nas atividades e entregáveis cruciais para o desenvolvimento.

\subsection{Cronograma de Atividades – Primeiro Semestre}

O desenvolvimento do sistema de gerenciamento de estoque será realizado ao longo de dois semestres letivos, estando o presente cronograma correspondente às atividades planejadas para o primeiro semestre. Neste período inicial, o foco principal está na fundamentação e planejamento do projeto, bem como na construção da prova de conceito e definição do escopo mínimo viável.

\begin{itemize}

\item No início do semestre, tem-se a fase de Desenvolvimento do Tema (Entrega do marco dia 08/04/2025), que contempla a escolha do assunto central do projeto e a definição do parceiro institucional. Essa etapa foca no levantamento detalhado dos requisitos (funcionais e não funcionais) e na análise de viabilidade do sistema proposto.

\item Posteriormente, desenvolve-se a fase de Desenho da Aplicação (Entrega do marco dia 29/04/2025), com foco na estruturação técnica da solução. Aqui, são entregues o projeto do banco de dados (MER e DER), os diagramas de arquitetura (UML) e a modelagem dos processos de negócio (AS IS e TO BE).

\item A seguir, passa-se à etapa de Prova de Conceito (Entrega do marco dia 20/05/2025), na qual se realiza a escolha das tecnologias mais adequadas. O resultado é o desenvolvimento dos protótipos iniciais de interface, dos scripts de criação do banco de dados e o início da implementação do \textit{back-end} do sistema.

\item Por fim, inicia-se a fase da Criação do MVP (Produto Mínimo Viável) (Entrega do marco dia 17/06/2025). Essa etapa inclui a definição clara do problema e das prioridades, delimitando o escopo da aplicação com suas restrições. O produto final é a identificação das funcionalidades mínimas essenciais e das integrações necessárias.

\item Paralelamente, é elaborada a fase de Análise e Documentação (Entrega do marco dia 18/07/2025), cujo objetivo é consolidar os dados obtidos. Esta seção inclui a elaboração da introdução, revisão da literatura, gestão do projeto e análise de viabilidade financeira, encerrando com as considerações finais sobre o processo.
\end{itemize}

% \subsection{Análise da duração do projeto}

\chapter{Viabilidade Financeira}

\section{Custos}

A Tabela 9 apresenta uma análise detalhada dos custos de desenvolvimento, com foco no investimento em mão de obra, que representa o principal ativo do projeto. A metodologia de cálculo é simples e transparente: o esforço previsto (em horas) é multiplicado pela taxa horária específica de cada especialista (desenvolvedores, analistas, gestores), garantindo uma apuração precisa e fiel aos valores de mercado.

Esta abordagem, portanto, transcende um simples levantamento numérico. Ela não só oferece clareza sobre os custos inerentes a cada fase do desenvolvimento, mas, ao ser projetada sobre a duração total do desenvolvimento, transforma-se em uma poderosa ferramenta de gestão estratégica e preditiva. A projeção temporal permite uma análise aprofundada da distribuição dos custos mensais e globais, oferecendo previsibilidade e permitindo a antecipação de picos de investimento. Com isso, a gestão pode mitigar riscos financeiros, como desvios orçamentários, antes que se concretizem.

Consequentemente, a tabela constitui um pilar para um planejamento financeiro robusto, que justifica o orçamento perante stakeholders e estabelece uma linha de base para o controle rigoroso das despesas, viabilizando a alocação otimizada de recursos e assegurando a sustentabilidade operacional do projeto.

\begin{table}[hbtp]
    \centering
    \caption{Detalhamento dos Custos de Desenvolvimento.}
    \label{tab:custos_desenvolvimento_simples}
    \begin{tabular}{|l|c|c|c|c|c|}
        \hline
        \textbf{Função} & \textbf{Quant.} & \textbf{Horas} & \textbf{Valor/Hora (R\$)} & \textbf{Total (R\$)} \\
        \hline
        Product Owner (PO)    & 1 & 400 & 35,00 & 14.000,00 \\
        \hline
        DBA     & 1 & 400 & 25,50 & 10.200,00 \\
        \hline
        Desenvolvedor Front-end & 3 & 250 & 18,00 & 13.500,00 \\
        \hline
        Desenvolvedor Back-end  & 3 & 250 & 20,00 & 15.000,00 \\
        \hline
        QA (Testes)           & 4 & 100 & 22,00 & 8.800,00 \\
        \hline
        \textbf{Total} & \textbf{7} & - & - & \textbf{61.500,00} \\
        \hline
    \end{tabular}
\fonte{Elaborado pelos autores.}
\end{table}

Na tabela 10 é representada a estrutura mensal da empresa, detalhando os custos associados às instalações, equipamentos e serviços de TI. No item 2.1, destacam-se os gastos com aluguel de espaço e mobiliário, totalizando 0 reais por mês. Já no item 2.2, referente à aquisição de equipamentos, observa-se que não há custos neste momento. No entanto, no item 2.3, estão listados os custos mensais de serviços de TI.

\begin{table}[hbtp]
\centering
\caption{Estrutura Mensal da Empresa.}
\label{tab:estrutura_mensal}
\begin{tabular}{lcr}
\toprule
\textbf{Item} & \textbf{Quantidade} & \textbf{Valor (R\$)} \\
\midrule
\multicolumn{3}{l}{\textbf{2. Estrutura da empresa (mensal)}} \\
\addlinespace % Adiciona um pequeno espaço vertical
\quad 2.1. Instalações - Aluguel & - & 0,00 \\
\addlinespace
\multicolumn{3}{l}{\quad 2.2. Equipamentos (TI e Outros)} \\
\qquad - Computadores/notebooks & 0 & 0,00 \\
\qquad - Servidores proprietários & 0 & 0,00 \\
\qquad - Outros (mobiliário, etc.) & 0 & 0,00 \\
\addlinespace
\multicolumn{3}{l}{\quad 2.3. Serviços de TI} \\
\qquad - Ferramentas de desenvolvimento & - & 0,00 \\
\qquad - Hospedagem e banco de dados & - &100,00 \\
\midrule
\textbf{Total mensal} & & \textbf{R\$ 100,00} \\
\bottomrule
\end{tabular}
\fonte{Elaborado pelos autores.}
\end{table}




\chapter{Considerações Finais}
Ao longo do desenvolvimento deste projeto, que visa a criação de um sistema de controle de estoque para um estabelecimento comercial, foi possível constatar de soluções tecnológicas para otimizar processos e garantir a eficiência operacional em ambientes comerciais. A iniciativa de desenvolver este sistema surgiu da necessidade de modernizar a gestão  em um cenário de mercado cada vez mais competitivo.
O sistema proposto representa uma ferramenta estratégica para a loja, Vip Penha permitindo o cadastro detalhado de produtos, o controle preciso de entradas e saídas de mercadorias, e a geração de relatórios gerenciais. Essas funcionalidades são necesarias para reduzir erros manuais, tempo dedicado à organização do estoque e, consequentemente, aprimorar a tomada de decisões. A implementação de uma metodologia ágil, como o Kanban, foi de grande ajuda para a organização do projeto.
Conclui-se que a aplicação deste sistema resulta em uma melhora substancial na gestão do estoque do estabelecimento, contribuindo para a redução de perdas, a otimização do fluxo de trabalho e por consequência ganhos econômicos. Além disso, este projeto demonstra o potencial das tecnologias da informação para resolver desafios práticos do cotidiano de empreendimentos. A contínua evolução e aprimoramento de tais sistemas é essencial para que empreendimentos e empresas   possam manter sua competitividade e prosperar no mercado atual.

%\section{Dificuldades, escolhas e descartes}
%Durante o desenvolvimento do projeto, enfrentamos desafios inerentes à complexidade de um sistema de controle de estoque, nas etapas iniciais de modelagem e arquitetura. Uma das primeiras dificuldades surgiu na elaboração do Diagrama de Entidade-Relacionamento (DER) e do diagrama de componentes. A primeira tentativa de construir esses diagramas revelou a necessidade de um aprofundamento maior das relações entre as entidades do sistema e na definição clara dos módulos e suas interações. Este processo inicial de tentativa foi crucial para refinar a visão e garantir uma base sólida para o desenvolvimento 
%Uma escolha crucial foi unificar o histórico de movimentações de produtos em uma única tabela, em vez de usar tabelas separadas para entradas e saídas. Essa decisão visou simplificar e otimizar a consulta e a unificação dos dados, melhorando o desempenho e a consistência.

% ---
% Apêndices
% ----------------------------------------------------------

% ---
% Inicia os apêndices
% ---
\begin{apendicesenv}

% Imprime uma página indicando o início dos apêndices
%\partapendices

% ----------------------------------------------------------
\chapter{Mapeamento de Processos}
% ----------------------------------------------------------
\section{Processo AS IS e TO BE}

\subsection{Processo AS IS (Situação Atual)}

\subsection*{Descrição Geral}
A loja faz controle de estoque manualmente, por anotações em papel feitas pelo dono.

\subsection*{Etapas do Processo}

\begin{enumerate}
    \item \textbf{Recebimento de produtos (entrada)}: O dono recebe os produtos de fornecedores. Ele anota em um caderno: nome do produto, quantidade recebida, data e fornecedor.
    
    \item \textbf{Venda de produtos (saída)}: Ao vender um produto, o dono anota à mão no mesmo caderno: nome do produto, quantidade vendida, data e valor da venda.
    
    \item \textbf{Controle de estoque}: O dono consulta o caderno para verificar a quantidade de produtos disponíveis. Não há atualização automática; o controle depende da leitura e interpretação das anotações.
    
    \item \textbf{Relatório mensal}: O dono folheia o caderno, soma manualmente as entradas e saídas e tenta fazer um resumo no final do mês. Processo demorado e sujeito a erros.
\end{enumerate}

\subsection*{Problemas Identificados}

\begin{itemize}
    \item Alto risco de erro humano.
    \item Dificuldade em rastrear movimentações específicas.
    \item Falta de relatórios precisos.
    \item Perda ou danos ao caderno comprometem todo o controle.
    \item Nenhuma visibilidade em tempo real do estoque.
\end{itemize}

\subsection{Processo TO BE (Situação Proposta com Sistema)}

\subsection*{Descrição Geral}
Será implementado um sistema informatizado para registrar entradas, saídas e gerar relatórios mensais automaticamente.

\subsection*{Etapas do Processo com o Sistema}

\begin{enumerate}
    \item \textbf{Recebimento de produtos (entrada)}: O dono ou funcionário acessa o sistema. Registra os produtos recebidos: nome, quantidade, fornecedor, data, nota fiscal. O estoque é atualizado automaticamente.

    \item \textbf{Venda de produtos (saída)}: No momento da venda, o produto é registrado como “vendido” no sistema. O sistema subtrai automaticamente a quantidade vendida do estoque.

    \item \textbf{Controle de estoque}: O sistema exibe em tempo real o estoque disponível de cada item. Alertas de baixo estoque podem ser configurados.

    \item \textbf{Relatório mensal de movimentações}: Ao fim do mês (ou sob demanda), o sistema gera automaticamente um relatório com:
    \begin{itemize}
        \item Total de entradas por produto.
        \item Total de saídas por produto.
        \item Saldo atual.
        \item Produtos mais vendidos.
        \item Histórico por período.
    \end{itemize}
\end{enumerate}

\subsection*{Benefícios Esperados}

\begin{itemize}
    \item Redução de erros manuais.
    \item Facilidade no rastreamento de movimentações.
    \item Economia de tempo.
    \item Maior controle e visibilidade sobre o estoque.
    \item Dados organizados e acessíveis a qualquer momento.
\end{itemize}
% ----------------------------------------------------------
%\lipsum[55-57]

\end{apendicesenv}
% ---
% ----------------------------------------------------------
% Finaliza a parte no bookmark do PDF
% para que se inicie o bookmark na raiz
% e adiciona espaço de parte no Sumário
% ----------------------------------------------------------
\phantompart




% ----------------------------------------------------------
% ELEMENTOS PÓS-TEXTUAIS
% ----------------------------------------------------------
\postextual
% ----------------------------------------------------------

% ----------------------------------------------------------
% Referências bibliográficas
% ----------------------------------------------------------
\bibliographystyle{abntex2-alf}
\bibliography{referencias}

% ----------------------------------------------------------
% Glossário
% ----------------------------------------------------------
%
% Consulte o manual da classe abntex2 para orientações sobre o glossário.
%
%\glossary

% ----------------------------------------------------------

% ----------------------------------------------------------
% Anexos
% ----------------------------------------------------------

% ---
% Inicia os anexos
% ---
%\begin{anexosenv}

% Imprime uma página indicando o início dos anexos
%\partanexos

% ---
%\chapter{exemplo 2}
% ---
%\lipsum[30]

% ---
%\chapter{Cras non urna sed feugiat cum sociis natoque penatibus et magnis dis
%parturient montes nascetur ridiculus mus}
% ---

%\lipsum[31]

% ---
%\chapter{exemplo 3}
% ---

%\lipsum[32]

%\end{anexosenv}

%---------------------------------------------------------------------
% INDICE REMISSIVO
%---------------------------------------------------------------------
\phantompart
\printindex
%---------------------------------------------------------------------

\end{document}
